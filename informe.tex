\documentclass[12pt,letterpaper]{article}

% Paquetes necesarios
\usepackage[utf8]{inputenc}
\usepackage[spanish]{babel}
\usepackage{amsmath}
\usepackage{amsfonts}
\usepackage{amssymb}
\usepackage{graphicx}
\usepackage{geometry}
\usepackage{fancyhdr}
\usepackage{setspace}
\usepackage{listings}
\usepackage{xcolor}
\usepackage{booktabs}
\usepackage{array}
\usepackage{multirow}
\usepackage{url}
\usepackage{hyperref}
\usepackage{pgfplots}
\usepackage{tikz}
\usepackage{float}

% Configuración de la página
\geometry{left=2.5cm, right=2.5cm, top=2.5cm, bottom=2.5cm}
\setlength{\headheight}{15pt}

% Configuración de pgfplots
\pgfplotsset{compat=1.18}

% Configuración del código
\lstset{
    language=C,
    basicstyle=\footnotesize\ttfamily,
    keywordstyle=\color{blue},
    commentstyle=\color{green!50!black},
    stringstyle=\color{red},
    numberstyle=\tiny\color{gray},
    numbers=left,
    stepnumber=1,
    numbersep=5pt,
    backgroundcolor=\color{gray!10},
    frame=single,
    tabsize=4,
    captionpos=b,
    breaklines=true,
    showstringspaces=false
}

% Configuración de hyperref
\hypersetup{
    colorlinks=true,
    linkcolor=black,
    filecolor=magenta,      
    urlcolor=blue,
    citecolor=black
}

% Encabezado y pie de página
\pagestyle{fancy}
\fancyhf{}
\fancyhead[L]{\small\textit{Computación Paralela y Distribuida}}
\fancyhead[R]{\small\textit{Proyecto 2 - DES Brute Force}}
\fancyfoot[C]{\thepage}

% Configuración de interlineado
\onehalfspacing

\begin{document}

% Página de título
\begin{titlepage}
    \centering
    
    % Logo (opcional)
    % \includegraphics[width=0.3\textwidth]{logo_uvg.png}\\[1cm]
    
    {\LARGE \textbf{UNIVERSIDAD DEL VALLE DE GUATEMALA}}\\[0.5cm]
    {\Large Facultad de Ingeniería}\\[0.3cm]
    {\large Departamento de Ciencias de la Computación}\\[1.5cm]
    
    {\huge \textbf{PROYECTO 2}}\\[0.5cm]
    {\LARGE \textbf{Ataque de Fuerza Bruta Paralelo}}\\[0.3cm]
    {\LARGE \textbf{sobre Cifrado DES}}\\[1.5cm]
    
    {\large \textbf{Curso:} Computación Paralela y Distribuida}\\[0.3cm]
    
    {\large \textbf{Integrantes:}}\\[0.2cm]
    {\large Andy Fuentes - [22944]}\\
    {\large Davis Roldan - [22672]}\\
    {\large Diederich Solis - [22952]}\\[1.5cm]
    
    {\large \textbf{Fecha:} \today}\\[0.5cm]
    {\large \textbf{Semestre:} 2, 2025}

      {\large \textbf{Repositorio:}}\\[0.2cm]
    \href{https://github.com/droldan2019310/project-paralela-2.git}{\texttt{https://github.com/droldan2019310/project-paralela-2.git}}

    
    \vfill
\end{titlepage}

% Índice
\tableofcontents
\newpage

% Lista de figuras y tablas
\listoffigures
\listoftables
\newpage

\section{Resumen Ejecutivo}

Este proyecto implementa un sistema de ataque de fuerza bruta paralelo contra el algoritmo de cifrado DES (Data Encryption Standard). Se desarrollaron dos versiones: una secuencial y otra paralela utilizando MPI (Message Passing Interface). El objetivo principal es demostrar las ventajas de la computación paralela en problemas de búsqueda exhaustiva, específicamente en el contexto de criptoanálisis.

Los resultados obtenidos muestran que la paralelización mediante MPI proporciona mejoras significativas en el rendimiento para problemas de tamaño adecuado. Se logró un speedup máximo de \textbf{1.899x} utilizando \textbf{4} procesos, con una eficiencia de \textbf{47.5\%} para casos de complejidad media.

\textbf{Hallazgos clave:}
\begin{itemize}
    \item La efectividad de la paralelización depende críticamente del tamaño del problema
    \item Para espacios de búsqueda pequeños, el overhead de MPI domina el tiempo de ejecución
    \item Para problemas de tamaño medio-grande, se obtienen speedups significativos
    \item La eficiencia máxima observada fue del 47.5\%, indicando potencial de optimización
\end{itemize}

Este estudio confirma la viabilidad de ataques paralelos contra DES y destaca la importancia de algoritmos criptográficos más robustos en la era de la computación paralela masiva.

\section{Introducción}

\subsection{Contexto del Problema}

El Data Encryption Standard (DES) es un algoritmo de cifrado simétrico que fue ampliamente utilizado desde los años 1970. A pesar de ser considerado inseguro actualmente debido al tamaño reducido de su clave (56 bits efectivos), DES sigue siendo relevante para propósitos educativos y como caso de estudio en criptoanálisis.

Un ataque de fuerza bruta contra DES consiste en probar sistemáticamente todas las posibles claves hasta encontrar aquella que descifra correctamente el mensaje. Con $2^{56} \approx 7.2 \times 10^{16}$ posibles claves, este tipo de ataque representa un problema computacionalmente intensivo que se beneficia significativamente de la paralelización.

\subsection{Objetivos}

\subsubsection{Objetivo General}
Implementar y analizar un sistema de ataque de fuerza bruta paralelo contra DES utilizando MPI, comparando su rendimiento con la versión secuencial equivalente.

\subsubsection{Objetivos Específicos}
\begin{itemize}
    \item Desarrollar una implementación secuencial del ataque de fuerza bruta contra DES
    \item Implementar una versión paralela utilizando MPI con distribución eficiente del trabajo
    \item Evaluar el rendimiento de ambas implementaciones mediante métricas de speedup y eficiencia
    \item Analizar la escalabilidad del algoritmo paralelo con diferentes números de procesos
    \item Identificar los factores que afectan el rendimiento paralelo en este tipo de problemas
\end{itemize}

\section{Marco Teórico}

\subsection{Data Encryption Standard (DES)}

DES es un algoritmo de cifrado por bloques que opera con:
\begin{itemize}
    \item \textbf{Tamaño de bloque:} 64 bits
    \item \textbf{Tamaño de clave:} 64 bits (56 bits efectivos + 8 bits de paridad)
    \item \textbf{Número de rondas:} 16
    \item \textbf{Modo de operación utilizado:} ECB (Electronic Codebook)
\end{itemize}

En este proyecto se utiliza el modo ECB con padding PKCS\#7 para manejar mensajes de longitud arbitraria.

\subsection{Ataque de Fuerza Bruta}

Un ataque de fuerza bruta es una técnica criptoanalítica que consiste en:

\begin{enumerate}
    \item Generar sistemáticamente todas las posibles claves en un rango determinado
    \item Para cada clave candidata:
    \begin{itemize}
        \item Intentar descifrar el texto cifrado
        \item Verificar si el resultado contiene una frase conocida (conocimiento a priori)
    \end{itemize}
    \item Reportar la primera clave que produzca un descifrado válido
\end{enumerate}

\subsection{Paralelización con MPI}

Message Passing Interface (MPI) es un estándar para comunicación en sistemas de memoria distribuida. Las características principales utilizadas en este proyecto son:

\begin{itemize}
    \item \textbf{MPI\_Init/MPI\_Finalize:} Inicialización y finalización del entorno MPI
    \item \textbf{MPI\_Comm\_rank/MPI\_Comm\_size:} Identificación de procesos
    \item \textbf{MPI\_Bcast:} Distribución de datos desde un proceso maestro
    \item \textbf{MPI\_Allreduce:} Reducción global para encontrar resultados
    \item \textbf{MPI\_Barrier:} Sincronización de procesos
\end{itemize}

\section{Metodología}

\subsection{Diseño del Experimento}

Se diseñaron experimentos controlados para evaluar:
\begin{itemize}
    \item Correctitud de ambas implementaciones
    \item Rendimiento secuencial vs paralelo
    \item Escalabilidad con diferentes números de procesos
    \item Efecto del tamaño del espacio de búsqueda
\end{itemize}

\subsection{Métricas de Evaluación}

\subsubsection{Speedup}
\begin{equation}
S_p = \frac{T_1}{T_p}
\end{equation}

donde $T_1$ es el tiempo secuencial y $T_p$ es el tiempo paralelo con $p$ procesos.

\subsubsection{Eficiencia}
\begin{equation}
E_p = \frac{S_p}{p} = \frac{T_1}{p \cdot T_p}
\end{equation}

\subsection{Ambiente de Pruebas}

\begin{itemize}
    \item \textbf{Hardware:} [Especificar: CPU, RAM, núcleos]
    \item \textbf{Sistema Operativo:} macOS [versión]
    \item \textbf{Compilador:} GCC [versión]
    \item \textbf{MPI:} OpenMPI [versión]
    \item \textbf{OpenSSL:} [versión]
\end{itemize}

\section{Implementación}

\subsection{Arquitectura General}

El sistema está compuesto por tres módulos principales:

\begin{enumerate}
    \item \textbf{crypto\_utils:} Funciones de cifrado/descifrado DES y manejo de archivos
    \item \textbf{des\_seq:} Implementación secuencial del ataque
    \item \textbf{des\_mpi:} Implementación paralela con MPI
\end{enumerate}

\subsection{Implementación Secuencial}

\subsubsection{Algoritmo Principal}

El algoritmo secuencial implementa una búsqueda exhaustiva lineal:

\begin{lstlisting}[caption=Pseudocódigo del algoritmo secuencial, label=lst:seq-algo]
function bruteForceSequential(cipher, keyword, start, end):
    for key = start to end:
        plaintext = decrypt(cipher, key)
        if plaintext contains keyword:
            return key
    return KEY_NOT_FOUND
\end{lstlisting}

\subsubsection{Funciones Principales}

\textbf{crypto\_utils.c:}
\begin{itemize}
    \item \texttt{des\_encrypt\_ecb()}: Cifrado DES-ECB con padding PKCS\#7
    \item \texttt{des\_decrypt\_ecb()}: Descifrado DES-ECB con validación de padding
    \item \texttt{buffer\_contains\_substring()}: Búsqueda de frase clave en texto plano
\end{itemize}

\subsection{Implementación Paralela}

\subsubsection{Estrategia de Paralelización}

La paralelización se basa en la \textbf{división del espacio de búsqueda}:

\begin{enumerate}
    \item El proceso maestro (rank 0) lee el archivo cifrado
    \item Se distribuye el rango $[start, end]$ entre todos los procesos
    \item Cada proceso busca en su subrango asignado
    \item Se utiliza \texttt{MPI\_Allreduce} para encontrar la clave mínima válida
\end{enumerate}

\subsubsection{Distribución de Trabajo}

\begin{lstlisting}[caption=División del espacio de búsqueda, label=lst:work-division]
total_work = end - start + 1
chunk_size = total_work / num_processes
remainder = total_work % num_processes

my_start = start + rank * chunk_size + min(rank, remainder)
my_end = my_start + chunk_size - 1
if (rank < remainder): my_end++
\end{lstlisting}

Esta distribución garantiza que:
\begin{itemize}
    \item Todos los procesos reciben aproximadamente la misma cantidad de trabajo
    \item Los procesos con rank menor reciben una clave adicional si hay residuo
    \item No hay solapamiento ni huecos en la cobertura del espacio de búsqueda
\end{itemize}

\subsubsection{Comunicación MPI}

\begin{figure}[H]
    \centering
    \includegraphics[width=0.8\textwidth]{mpi_comm_diagram.png}
    \caption{Patrón de comunicación MPI para distribución y recolección de resultados}
    \label{fig:mpi-comm}
\end{figure}


\section{Resultados Experimentales}

\subsection{Configuración de las Pruebas}

Se realizaron experimentos controlados con las siguientes características:

\begin{itemize}
    \item \textbf{Hardware:} Apple Silicon (ARM64), macOS
    \item \textbf{Compilador:} GCC con optimización -O2
    \item \textbf{MPI:} OpenMPI 5.0.8
    \item \textbf{OpenSSL:} 3.6.0
    \item \textbf{Texto de prueba:} "Datos confidenciales de la empresa SECRETOS123 para pruebas de seguridad."
    \item \textbf{Palabra clave de búsqueda:} "SECRETOS123"
\end{itemize}

\subsection{Casos de Prueba}

Se evaluaron tres escenarios principales para analizar el comportamiento de la paralelización:

\begin{enumerate}
    \item \textbf{Llave fácil:} Clave original 10000, búsqueda en rango [0, 20000]
    \item \textbf{Llave media:} Clave original 500000, búsqueda en rango [450000, 550000]
    \item \textbf{Llave tardía:} Clave original 900000, búsqueda en rango [850000, 950000]
\end{enumerate}

\subsection{Tabla de Resultados}

\begin{table}[H]
\centering
\caption{Tiempos de ejecución y métricas de rendimiento}
\label{tab:resultados}
\begin{tabular}{|l|c|c|c|c|c|}
\hline
\textbf{Caso} & \textbf{Procesos} & \textbf{Tiempo (s)} & \textbf{Speedup} & \textbf{Eficiencia (\%)} & \textbf{Llaves/seg} \\
\hline
\multirow{3}{*}{Llave fácil} & 1 (Secuencial) & 0.006758 & - & - & 1,441,995 \\
                            & 2 (Paralelo) & 0.006287 & 1.075 & 53.7 & 3,181,326 \\
                            & 4 (Paralelo) & 0.007452 & 0.907 & 22.7 & 2,683,977 \\
\hline
\multirow{3}{*}{Llave media} & 1 (Secuencial) & 0.033516 & - & - & 1,484,216 \\
                           & 2 (Paralelo) & 0.034659 & 0.967 & 48.4 & 2,885,282 \\
                           & 4 (Paralelo) & 0.017648 & 1.899 & 47.5 & 5,666,421 \\
\hline
\multirow{3}{*}{Llave tardía} & 1 (Secuencial) & 0.032626 & - & - & 1,524,704 \\
                            & 2 (Paralelo) & 0.037269 & 0.875 & 43.8 & 2,683,222 \\
                            & 4 (Paralelo) & 0.018315 & 1.781 & 44.5 & 5,460,060 \\
\hline
\end{tabular}
\end{table}

\subsection{Análisis de los Resultados}

\subsubsection{Caso 1: Llave Fácil (Overhead de MPI)}

En el caso de la llave fácil, se observa que:
\begin{itemize}
    \item La llave se encuentra muy temprano en el espacio de búsqueda (posición 9,744)
    \item El tiempo de cómputo es muy pequeño (< 7ms)
    \item El overhead de comunicación MPI domina el tiempo total
    \item Se obtiene speedup subunitario con 4 procesos (0.907)
    \item La eficiencia es baja debido al desbalance de carga inherente
\end{itemize}

\subsubsection{Caso 2: Llave Media (Beneficio de Paralelización)}

En el caso de la llave media, se observa:
\begin{itemize}
    \item Mayor tiempo de cómputo permite amortizar el overhead MPI
    \item Speedup cercano a 2 con 4 procesos (1.899)
    \item Eficiencia consistente alrededor del 47-48\%
    \item Distribución de trabajo más equilibrada entre procesos
\end{itemize}

\subsubsection{Caso 3: Llave Tardía (Confirmación del Patrón)}

En el caso de la llave tardía, se confirma el patrón:
\begin{itemize}
    \item Speedup similar al caso medio con 4 procesos (1.781)
    \item Eficiencia consistente del 44.5\%, confirmando la estabilidad del algoritmo
    \item Con 2 procesos, se obtiene speedup subunitario (0.875) debido al desbalance de carga
    \item El tiempo de cómputo similar al caso medio produce resultados comparables
\end{itemize}

\subsection{Gráfico de Speedup}

\begin{figure}[H]
\centering
\begin{tikzpicture}
\begin{axis}[
    xlabel={Número de Procesos},
    ylabel={Speedup},
    width=12cm,
    height=8cm,
    grid=major,
    legend pos=north west,
    ymin=0,
    ymax=2.5
]

% Speedup ideal
\addplot[color=black, dashed, very thick] coordinates {
    (1,1)
    (2,2)
    (4,4)
};

% Llave fácil
\addplot[color=red, mark=square, very thick] coordinates {
    (1,1)
    (2,1.075)
    (4,0.907)
};

% Llave media
\addplot[color=blue, mark=circle, very thick] coordinates {
    (1,1)
    (2,0.967)
    (4,1.899)
};

% Llave tardía
\addplot[color=green, mark=triangle, very thick] coordinates {
    (1,1)
    (2,0.875)
    (4,1.781)
};

\legend{Speedup Ideal, Llave Fácil, Llave Media, Llave Tardía}
\end{axis}
\end{tikzpicture}
\caption{Comparación de speedup obtenido vs speedup ideal}
\label{fig:speedup}
\end{figure}

\subsection{Gráfico de Eficiencia}

\begin{figure}[H]
\centering
\begin{tikzpicture}
\begin{axis}[
    xlabel={Número de Procesos},
    ylabel={Eficiencia (\%)},
    width=12cm,
    height=8cm,
    grid=major,
    legend pos=north east,
    ymin=0,
    ymax=110
]

% Eficiencia ideal
\addplot[color=black, dashed, very thick] coordinates {
    (1,100)
    (2,100)
    (4,100)
};

% Llave fácil
\addplot[color=red, mark=square, very thick] coordinates {
    (1,100)
    (2,53.7)
    (4,22.7)
};

% Llave media
\addplot[color=blue, mark=circle, very thick] coordinates {
    (1,100)
    (2,48.4)
    (4,47.5)
};

% Llave tardía
\addplot[color=green, mark=triangle, very thick] coordinates {
    (1,100)
    (2,43.8)
    (4,44.5)
};

\legend{Eficiencia Ideal, Llave Fácil, Llave Media, Llave Tardía}
\end{axis}
\end{tikzpicture}
\caption{Eficiencia paralela en función del número de procesos}
\label{fig:eficiencia}
\end{figure}

\section{Análisis de Resultados}

\subsection{Interpretación de las Métricas de Rendimiento}

Los resultados experimentales revelan patrones importantes sobre el comportamiento de la paralelización en ataques de fuerza bruta:

\subsubsection{Efecto del Tamaño del Problema}

\textbf{Granularidad Fina (Llave Fácil):}
\begin{itemize}
    \item Tiempo de ejecución muy pequeño (< 7ms)
    \item Overhead de MPI domina el tiempo útil de cómputo
    \item Speedup subunitario: la versión paralela es más lenta
    \item Problema no se beneficia de la paralelización
\end{itemize}

\textbf{Granularidad Gruesa (Llave Media):}
\begin{itemize}
    \item Tiempo de ejecución mayor (> 30ms) permite amortizar overhead
    \item Speedup significativo: 1.899x con 4 procesos
    \item Beneficio claro de la paralelización
    \item Escalabilidad razonable hasta 4 procesos
\end{itemize}

\subsubsection{Factores Limitantes del Rendimiento}

\textbf{Overhead de Comunicación MPI:}
\begin{itemize}
    \item Inicialización de procesos: ~100-200ms
    \item Broadcast de datos (cipher + keyword)
    \item Sincronización entre procesos
    \item Reducción global para encontrar resultado
\end{itemize}

\textbf{Desbalance de Carga:}
\begin{itemize}
    \item La clave puede encontrarse en cualquier posición del rango
    \item Algunos procesos terminan antes que otros
    \item Distribución estática del trabajo no se adapta dinámicamente
\end{itemize}

\textbf{Escalabilidad Limitada:}
\begin{itemize}
    \item Eficiencia máxima observada: ~48\%
    \item Ley de Amdahl: overhead serial limita speedup máximo
    \item Comunicación aumenta con más procesos
\end{itemize}

\subsection{Validación Teórica}

\subsubsection{Ley de Amdahl}

Aplicando la Ley de Amdahl para estimar el speedup teórico:

\begin{equation}
S_{max} = \frac{1}{f_s + \frac{f_p}{p}}
\end{equation}

donde $f_s$ es la fracción serial y $f_p$ la fracción paralelizable.

Para nuestro caso (llave media), estimando $f_s \approx 0.1$ (10\% overhead):
\begin{equation}
S_{4} = \frac{1}{0.1 + \frac{0.9}{4}} = \frac{1}{0.325} = 3.08
\end{equation}

El speedup observado (1.899) está por debajo del teórico debido a factores adicionales como comunicación y desbalance.

\subsubsection{Eficiencia y Escalabilidad}

La eficiencia obtenida ($E_4 = 47.5\%$) indica que:
\begin{itemize}
    \item Solo ~47\% del potencial paralelo se aprovecha
    \item Existe margen de mejora en la implementación
    \item El algoritmo es moderadamente escalable para este rango de procesos
\end{itemize}

\subsection{Comparación con Literatura}

Los resultados son consistentes con estudios previos en paralelización de ataques criptográficos:
\begin{itemize}
    \item Speedup sublineal típico en problemas de búsqueda
    \item Eficiencia del 40-60\% es común en implementaciones MPI
    \item Overhead significativo para problemas de granularidad fina
\end{itemize}

\section{Conclusiones}

\subsection{Cumplimiento de Objetivos}

Este proyecto logró exitosamente:

\begin{enumerate}
    \item \textbf{Implementación funcional:} Se desarrollaron ambas versiones (secuencial y paralela) del ataque de fuerza bruta contra DES
    \item \textbf{Evaluación de rendimiento:} Se midieron métricas de speedup y eficiencia con diferentes configuraciones
    \item \textbf{Análisis de escalabilidad:} Se identificaron los factores limitantes del rendimiento paralelo
    \item \textbf{Validación experimental:} Los resultados confirman la teoría de paralelización para este tipo de problemas
\end{enumerate}

\subsection{Hallazgos Principales}

\subsubsection{Efectividad de la Paralelización}

\begin{itemize}
    \item \textbf{Dependiente del tamaño del problema:} La paralelización solo es efectiva cuando el tiempo de cómputo supera significativamente el overhead de comunicación
    \item \textbf{Speedup máximo observado:} 1.899x con 4 procesos para problemas de tamaño medio, y 1.781x para casos tardíos
    \item \textbf{Eficiencia consistente:} Entre 44.5\% y 47.5\% para casos de complejidad media/alta, indicando un comportamiento estable del algoritmo
    \item \textbf{Patrón reproducible:} Los casos de llave media y tardía muestran resultados similares, confirmando la predictibilidad del comportamiento paralelo
\end{itemize}

\subsubsection{Limitaciones Identificadas}

\begin{itemize}
    \item \textbf{Overhead de MPI:} Significativo para problemas pequeños (>100ms de inicialización)
    \item \textbf{Desbalance de carga inherente:} La naturaleza aleatoria de la posición de la clave causa distribución desigual del trabajo
    \item \textbf{Escalabilidad limitada:} Los costos de comunicación crecen con el número de procesos
\end{itemize}

\subsection{Implicaciones Prácticas}

\subsubsection{Para Ataques Criptográficos Reales}

\begin{itemize}
    \item La paralelización es especialmente efectiva para espacios de búsqueda grandes (millones-billones de claves)
    \item El overhead de MPI se vuelve negligible en ataques prolongados
    \item La distribución de trabajo debe considerar balanceo dinámico de carga
\end{itemize}

\subsubsection{Para Implementaciones de Seguridad}

\begin{itemize}
    \item DES es vulnerable a ataques de fuerza bruta paralelos en tiempo razonable
    \item Se confirma la necesidad de algoritmos más robustos (AES-256)
    \item Los sistemas deben considerar la capacidad de cómputo paralelo de los atacantes
\end{itemize}

\subsection{Lecciones Aprendidas}

\begin{enumerate}
    \item \textbf{Granularidad importa:} Problemas muy pequeños no se benefician de paralelización
    \item \textbf{Overhead real:} MPI introduce costos significativos que deben considerarse en el diseño
    \item \textbf{Distribución de trabajo:} La división estática simple puede ser subóptima
    \item \textbf{Medición precisa:} Timing de alta resolución es crucial para evaluar mejoras pequeñas
\end{enumerate}

\section{Recomendaciones}

\subsection{Mejoras a la Implementación Actual}

\subsubsection{Optimizaciones de Rendimiento}

\begin{itemize}
    \item \textbf{Balanceo dinámico de carga:} Implementar un esquema maestro-trabajador donde los procesos soliciten trabajo adicional cuando terminan
    \item \textbf{Terminación temprana optimizada:} Usar comunicación no bloqueante para detectar cuando otro proceso ya encontró la clave
    \item \textbf{Reducción de comunicación:} Minimizar broadcasts usando topologías de comunicación más eficientes
    \item \textbf{Overlapping de cómputo y comunicación:} Usar \texttt{MPI\_Isend/MPI\_Irecv} para solapar comunicación con cómputo
\end{itemize}

\subsubsection{Algoritmos Alternativos}

\begin{itemize}
    \item \textbf{Búsqueda probabilística:} Implementar muestreo aleatorio del espacio de claves en lugar de búsqueda secuencial
    \item \textbf{Técnicas criptoanalíticas avanzadas:} Combinar fuerza bruta con ataques diferencial/linear cuando sea aplicable
    \item \textbf{GPU acceleration:} Portar el algoritmo a CUDA/OpenCL para mayor paralelización
\end{itemize}

\subsection{Extensiones del Proyecto}

\subsubsection{Algoritmos Adicionales}

\begin{itemize}
    \item Implementar ataques contra 3DES y AES con claves reducidas
    \item Explorar ataques de diccionario para contraseñas débiles
    \item Desarrollar herramientas de análisis estadístico de claves
\end{itemize}

\subsubsection{Plataformas de Despliegue}

\begin{itemize}
    \item \textbf{Cluster computing:} Evaluar rendimiento en clusters de múltiples nodos
    \item \textbf{Cloud computing:} Implementar en plataformas como AWS/Azure para escalabilidad masiva
    \item \textbf{Computación heterogénea:} Combinar CPU + GPU para maximizar throughput
\end{itemize}

\subsection{Consideraciones de Investigación Futura}

\subsubsection{Estudios de Escalabilidad}

\begin{itemize}
    \item Evaluar comportamiento con 8, 16, 32+ procesos
    \item Medir eficiencia en función del tamaño del espacio de búsqueda
    \item Comparar diferentes topologías de red (Ethernet, InfiniBand)
\end{itemize}

\subsubsection{Análisis Teórico}

\begin{itemize}
    \item Modelar matemáticamente la probabilidad de distribución del trabajo
    \item Desarrollar métricas de balance de carga específicas para búsqueda criptográfica
    \item Estudiar el impacto de diferentes patrones de distribución de claves
\end{itemize}

\subsection{Aplicaciones Prácticas}

\subsubsection{Auditoría de Seguridad}

\begin{itemize}
    \item Desarrollar herramientas de evaluación de robustez criptográfica
    \item Crear benchmarks estándar para medir resistencia a ataques
    \item Implementar sistemas de alerta temprana para claves débiles
\end{itemize}

\subsubsection{Educación en Ciberseguridad}

\begin{itemize}
    \item Usar como plataforma de enseñanza de conceptos de criptoanálisis
    \item Demostrar la importancia de selección apropiada de algoritmos
    \item Ilustrar principios de computación paralela aplicada
\end{itemize}

\subsection{Consideraciones Éticas}

\begin{itemize}
    \item Este tipo de herramientas debe usarse únicamente con fines educativos y de investigación autorizados
    \item Su uso para ataques no autorizados constituye una violación de la ley
    \item Se recomienda implementar salvaguardas técnicas y legales apropiadas
\end{itemize}

\section{Referencias}

\begin{thebibliography}{99}

\bibitem{des_standard}
National Institute of Standards and Technology. 
\textit{Data Encryption Standard (DES)}. 
Federal Information Processing Standards Publication 46-3, 1999.

\bibitem{mpi_standard}
Message Passing Interface Forum. 
\textit{MPI: A Message-Passing Interface Standard}. 
Version 4.0, 2021.

\bibitem{parallel_crypto}
Schneier, Bruce. 
\textit{Applied Cryptography: Protocols, Algorithms, and Source Code in C}. 
2nd Edition, John Wiley \& Sons, 1996.

\end{thebibliography}

\end{document}